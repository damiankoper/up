\documentclass[12pt]{article}
\usepackage{scrextend}
\usepackage[utf8]{inputenc}
\usepackage[polish]{babel}
\usepackage[T1]{fontenc}%polskie znaki
\usepackage[utf8]{inputenc}%polskie znaki
\usepackage{geometry}
\usepackage{float}
\usepackage{enumitem}
\usepackage{hyperref}
\usepackage{graphicx}
\usepackage{tabulary}
\usepackage{etoc}
\usepackage[normalem]{ulem} 
\usepackage{tikz}
\usepackage[bf]{caption}
\renewcommand{\baselinestretch}{1.5}

\usepackage{listings}
\usepackage{xcolor}
 
\definecolor{codegreen}{rgb}{0,0.6,0}
\definecolor{codegray}{rgb}{0.5,0.5,0.5}
\definecolor{codepurple}{rgb}{0.58,0,0.82}
\definecolor{backcolour}{rgb}{0.95,0.95,0.92}
 
\lstdefinestyle{mystyle}{
    backgroundcolor=\color{backcolour},   
    commentstyle=\color{codegreen},
    keywordstyle=\color{magenta},
    numberstyle=\tiny\color{codegray},
    stringstyle=\color{codepurple},
    basicstyle=\ttfamily\footnotesize,
    breakatwhitespace=false,         
    breaklines=true,                 
    captionpos=b,                    
    keepspaces=true,                 
    numbers=left,                    
    numbersep=5pt,                  
    showspaces=false,                
    showstringspaces=false,
    showtabs=false,                  
    tabsize=2
}
\renewcommand{\lstlistlistingname}{Spis listingów}\lstset{style=mystyle}

\graphicspath{ {img/} }
\newgeometry{lmargin=2.0cm, rmargin=2.0cm, tmargin=2.0cm, bmargin=2.0cm}
\clubpenalty=9996
\widowpenalty=9999
\brokenpenalty=4991
\predisplaypenalty=10000
\postdisplaypenalty=1549
\displaywidowpenalty=1602

\title{ 
    \vspace*{55mm}
    \textsc{
        \textbf{Urządzenia Peryferyjne}\\
        \large Sprawozdanie  \\
        \Large Sterowaniem silnikiem krokowym za pomocą USB
        }
} 
\author{
Damian Koper,  241292\\
Wiktor Pieklik, 241282\\
}

\date{\today}

\begin{document}

\maketitle

\newpage
\setcounter{tocdepth}{2}
\localtableofcontents
\listoffigures
\lstlistoflistings

\newpage

\section{Cel ćwiczenia}
% Co tam trzeba było zrobić
\section{Silnik krokowy}
% Ogólna definicja
\subsection{Budowa i zasada działania}
% Streszczenie tego z obrazkami
% http://silniki-krokowe.com.pl/informacje-techniczne/silniki-krokowe-zasada-dzialania-2/
\subsection{Sterowanie}
% Streszczenie tego z obrazkami
% http://silniki-krokowe.com.pl/informacje-techniczne/sterowanie-silnikami-krokowymi/#comment-1298
% + Soft-start
\section{Układ sterowania}
% Omówić ten obrazek
% http://www.zsk.ict.pwr.wroc.pl/zsk/repository/dydaktyka/up/lab/silnik_krokowy/schemat.gif
\subsection{Układ MMusb245}
% Co mniej więcej robi
\subsection{Układ ULN2803A}
% Co mniej więcej robi - działa jako wzmacniacz sygnału sterowania
\section{Sterowanie silnikiem}
% Tutaj można dodać więcej subsekszyn
% Omówienie biblioteki https://www.ftdichip.com/
% Screeny i omówienie probramu i jego opcji.
% Sekwencja impulsów(wskazać, że był błąd w kolejności)
% Wyznaczanie ilości cykli impulsów na obrót - czyli na oko
\end{document}